\documentclass[12pt]{article}
\usepackage[english]{babel}
\usepackage[utf8x]{inputenc}
\usepackage{amsmath}
\usepackage{hyperref}
\usepackage{graphicx}
\usepackage{float}
\usepackage[nottoc]{tocbibind}
\usepackage[colorinlistoftodos]{todonotes}
\setlength{\marginparwidth}{2cm}
\hypersetup{hidelinks}
\begin{document}

\begin{titlepage}

\newcommand{\HRule}{\rule{\linewidth}{0.5mm}} % Defines a new command for the horizontal lines, change thickness here

\center % Center everything on the page
 
%----------------------------------------------------------------------------------------
%	HEADING SECTIONS
%----------------------------------------------------------------------------------------

\textsc{\LARGE Università di Catania}\\[1.5cm] % Name of your university/college
\includegraphics[scale=.3]{logoUnict.png}\\[1cm] % Include a department/university logo - this will require the graphicx package
\textsc{\Large MRKT Tutorial}\\[0.5cm] % Major heading such as course name

%----------------------------------------------------------------------------------------
%	TITLE SECTION
%----------------------------------------------------------------------------------------

\HRule \\[0.4cm]
{ \huge \bfseries How to set up your environment for Hololens}\\[0.4cm] % Title of your document
\HRule \\[1.5cm]
 
%----------------------------------------------------------------------------------------
%	AUTHOR SECTION
%----------------------------------------------------------------------------------------

\begin{minipage}{0.4\textwidth}
\begin{flushleft} \large
\emph{Author:}\\
Marco \textsc{Ardizzone}\\ % Your name

\vspace{0.3cm}
\emph{Matricola:}\\
X81001077\\ % Your name
\end{flushleft}

\end{minipage}\\[2cm]

% If you don't want a supervisor, uncomment the two lines below and remove the section above
%\Large \emph{Author:}\\
%John \textsc{Smith}\\[3cm] % Your name

%----------------------------------------------------------------------------------------
%	DATE SECTION
%----------------------------------------------------------------------------------------

{\large May 2021}\\[2cm] % Date, change the \today to a set date if you want to be precise

\vfill % Fill the rest of the page with whitespace

\end{titlepage}



\tableofcontents
\clearpage
\section{Introduction}
\textbf{Disclaimer}: This tutorial is not meant to replace the \href{https://docs.microsoft.com/en-us/windows/mixed-reality/develop/unity/tutorials/mr-learning-base-01}{\underline{\emph{original MRTK tutorial}}}, it is just a tutorial made for other students which wants to quickly set up the environment for Microsoft Hololens. 
 
%i nomi dei pacchetti in inglese potrebbero essere leggermente diversi
%sarebbe meglio aggiungere delle immagini passo-passo
\section{Prerequisites}
Before starting, it is mandatory to download and set up the following resources:
\begin{itemize}
  \item \textbf{Windows 10}
  \item \textbf{Visual Studio 2019 (16.8 or higher)}: Downloadable \href{https://visualstudio.microsoft.com/downloads/}{\underline{\emph{here}}}.
  \\
  It is \emph{mandatory} to install these extensions:
  \subitem \textbf{ASP:NET and Web Development}
  \subitem \textbf{.NET desktop development}
  \subitem \textbf{C++ desktop development}
  \subitem \textbf{UWP development}
  \subitem \textbf{Unity game development}
  \subitem \textbf{C++ game development}
  \subitem \textbf{USB devices connectivity}
  \subitem \textbf{Windows 10 SDK}
  
  \item \textbf{Windows 10 SDK}: Downloadable \href{https://developer.microsoft.com//windows/downloads/windows-10-sdk}{\underline{\emph{here}}} 
  \item \textbf{Unity 2019.4 LTS}: Downloadable \href{https://unity3d.com/get-unity/download}{\underline{\emph{here}}}.
  \\
  It is \emph{mandatory} to install modules:
  \subitem \textbf{Universal Windows Platform Build Support}
  \subitem \textbf{Windows Build Support (IL2CPP)}
  
  \item \textbf{.NET 5.0 runtime (or higher)}: Downloadable \href{https://dotnet.microsoft.com/download/dotnet/5.0}{\underline{\emph{here}}}.
  
  \item \textbf{Mixed Reality ToolKit}: Downloadable \href{https://aka.ms/MRFeatureTool}{\underline{\emph{here}}}.
\end{itemize}

\section{Initializing and deploying your first application}
\subsection{Unity Project}
First of all, a new Unity 3D Project must be created.
Once the blank project is created, select\emph{File} $>$ \emph{Build Settings...}. \\
Then select \emph{Universal Windows Platform} and click \emph{Switch Platform}.
Once the switch is done, close the Build Settings Window. Open the \emph{Window} menu $>$ \emph{TextMeshPro} $>$ \emph{Import TMP Essential Resources}, then click \emph{all} and \emph{import}

\subsection{Importing the Mixed Reality ToolKit}
In order to import the Mixed Reality ToolKit in your Unity's Project, you must open the \emph{MixedRealityFeatureTool.exe} from the Mixed Reality Feature Tool directory.
Then, \emph{Start} $>$ select \emph{Mixed Reality ToolKit Fundation (2.5.4 or higher)} and click on \emph{Get Features}.
Click on \emph{...} and select your project's path, then click \emph{Validate}
You will get a popup, close it and click on \emph{Import} and then click on \emph{Approve}

\subsection{Configuring the Unity Project}
Once unity has finished importing the packages, click on \emph{Mixed Reality ToolKit} menu $>$ \emph{utilities} $>$ \emph{Configure Unity Project}.
In the Configuration Windows, expand \emph{Modify Configurations section}, ensure all options are checked and click on \emph{Apply}.
Now, select \emph{Edit} menu $>$ \emph{Project Settings...} and then \emph{XR Plug-in Management} $>$ \emph{Install XR Plug-in Management}. Once the installation is done, ensure that \emph{Initialize XR on Startup} and \emph{Windows Mixed Reality} is checked.
%a me sta cosa non andava, sarà un problema?
Then in MRTK Project Configurator window, use the \emph{Audio spatializer} dropdown to select the \emph{MS HRTF Spatializer}, then click the \emph{Apply} button.
Then in the Project Settings Window, select \emph{XR Plug-in Management} $>$ \emph{Windows Mixed Reality} $>$ \emph{Runtime Settings} and select $16 bit-depth$ as \emph{Depth Buffer Format}.
In project settings window, select \emph{Player} $>$ \emph{Publishing Settings} and in the \emph{Package name} enter a suitable name.




\section{Conclusion}
Once everything is done, you can write and build your application on hololens!


\end{document}